\chapter{Introduction}
In this internship my work can be divided into three main blocks, \ac{DNNDK} testing, \ac{DNNDK} demonstrator on Enclustra hardware and finally a collaboration project.

After investigating the possibilities of available \ac{ANN} accelerators and market research, the Xilinx \ac{DNNDK} was studied in more detail. The available examples were evaluated using the Xilinx ZCU 104 evaluation board. This was used as a test platform to check the published sample applications in the field of image recognition. The main application area is image classification and detection.

The next step was porting the available examples and use cases to Enclustra custom hardware. In addition, a demonstrator for \ac{ANN} applications was developed using the Enclustra Mars ST3 baseboard and Mars XU 3 \ac{FPGA} module. Image classification using resnet50 and face detection were implemented and tested. The whole design included using Vivado for the \ac{DPU} \ac{IP} core integration and then building an embedded Linux \ac{OS} with Xilinx Petalinux.

The other main part of the internship was developing a more sophisticated \ac{ANN} demonstrator in collaboration with Synthara. Synthara is an ETH spin-off and offers their own \ac{IP} core for neural network applications. The idea for the demonstrator was for a human player to play Rock-Paper-Scissors against a robot hand. The decision which symbol to choose would be done by using a camera and running a hand gesture detection neural network on the \ac{FPGA}. For this purpose, a data set was collected and a custom neural network trained. The hardware implementation was also started by using the Synthara neural network accelerator \ac{IP} core.

Throughout the internship several presentations were held as well. The topics ranged from introduction to \ac{AI} and \ac{ML} in general, to \ac{AI} on \acp{FPGA} and data set collection.