\chapter{Week}
At the beginning of the week the first presentation 'Introduction to \ac{AI}' was held before the technical stuff of the company. The general background and principles of \ac{ML} have been introduced and an outlook given to the second presentation, which would go more into detail about the actual hardware realization. The rest of the week was spent going through various tutorials provided by Xilinx to familiarize myself with the workflow and the \ac{DNNDK} toolkit. As the state of tools used for \ac{AI} applications on \ac{FPGA} is still in flux, several approaches needed to be evaluated:
\begin{itemize}
	\item \textbf{\ac{DNNDK} workflow}: Version 2.08 of the toolkit supported only the Caffe neural network training framework and needs a network description file and the trained weights as input. The key component here is the \ac{DPU} \ac{IP} core provided by the \ac{DNNDK} toolkit. This core is integrated via Vivado into the block design of the hardware and can be configured and adjusted for several performance and power profiles.
	\item \textbf{\ac{DNNDK} \ac{SDSoC}}: Another option is to abstract away the whole Vivado block design process and use Xilinx \ac{SDSoC} to implement the whole system in a higher programming language, C++. Supported functions can then be flagged as being executed in the \ac{PL} part of the system. This approach makes using a traditional \ac{HDL} obsolete and is deemed more accessible. This approach uses the established Xilinx reVision stack for development providing high level \acp{API} for computer vision.
\end{itemize}